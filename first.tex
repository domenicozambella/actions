% !TEX root = actions.tex
\section{Preliminaries}
\label{actions}

\def\ceq#1#2#3{\noindent\parbox[t]{15ex}{$\displaystyle #1$}\parbox{6ex}{\hfil $#2$}{$\displaystyle #3$}}


Let $\Delta\subseteq L_{x\,z}(\U)$.
Given any \emph{$\X$\/} $\subseteq\U^x$ and \emph{$\Z$\/} $\subseteq\U^z$, we define the structure \emph{$\UDelta$\/} $=\big\langle\X\,;\Z\big\rangle$.
This is a 2-sorted structure whose signature \emph{$\LDelta$\/} contains relation symbols for every formula $\varphi(x\,;z)\in\Delta$.
As there is little risk of confusion, these relations symbols are also denoted by $\varphi(x\,;z)$, and we identify $\LDelta$ with a subset of $L$.

When $\X$ and $\Z$ are type definable and $\Delta$ is small, $\UDelta$ is a saturated structure.
As compactness is not always required, these sets are arbitrary unless explicitly required.

In the first few sections we only work a subset of the quantifier-free fragment of $\LDelta$.
Let $\BDelta(\Z)$ be the set of Boolean combinations of the sets $\varphi(\X\,;b)$, for $\varphi(x\,;z)\in\Delta$ and $b\in\Z$.
The same symbol is also used to denote the collection of formulas defining these sets.
A $\BDelta(\Z)$-type is a collection of $\BDelta(\Z)$-formulas.
A $\BDelta(\Z)$-type-definable set is one defined by a $\BDelta(\Z)$-type of small cardinality.

We write \emph{$S_\Delta(\Z)$\/} for the set of $\BDelta(\Z)$-types that are maximallyfinitely consistent (in $\X$)~--~which we may conveniently identify with a sets of $\Delta(\Z)$ and negated $\Delta(\Z)$-formulas.
Note that the types in $S_\Delta(\Z)$ are required to be finitely consistent $\X$~--~a piece of information that we will not display in the notation.
When the types are required to be finitely consistent in some $\Y\subseteq\X$ we write \emph{$S_{\Delta,\Y}(\Z)$.}

% A set $\D\subseteq\X$ is \emph{$\BDelta({\Z})$-definable\/} if $\D=\varphi(\X)$ for some $\varphi(x)\in\BDelta({\Z})$.
% In the literature these sets are sometime called \textit{relatively\/} definable to distinguish them from $\varphi(\U^x)$ and $\varphi(\U^z)$.
% Similarly, we define \emph{$\BDelta({\Z})$-type-definable\/} sets.

When $p(x)$ a type (i.e.\@ any set of formulas) and $\D\subseteq\X$ we write $p(x)\vdash x\in\D$ if the inclusion $\psi(\X)\subseteq\D$ holds for some $\psi(x)$ that is conjunctions of formulas in $p(x)$.

Let \emph{$G$\/} $\le{\rm Aut}(\UDelta)$.
We view $G$ as a group acting on $\UDelta$ and write ${\cdot}$ for such action.
We write $g{\cdot}\D$ for the natural action of $g\in G$ on $\D\subseteq\X$.
If $\Delta=\varphi(\X\,;b)$ then $g{\cdot}\D=\varphi(\X\,;g{\cdot}b)$.
A formula is $G$-invariant if the set it defines is $G$-invariant (i.e.\@ fixed by the action of $G$).

If $p(x)$ is a $\BDelta(\Z)$-type, we write $g{\cdot}p(x)$ for the set of formulas of the form $\varphi(\X\,;g{\cdot}b)$ for $\varphi(\X\,;b)\in p$.
We say that the type $p(x)\subseteq\BDelta(\Z)$ is $G$-invariant accordingly.

We say that a set $\D\subseteq\X$ is \emph{syndetic\/} under the action of $G$, or \emph{$G$-syndetic\/} for short, if finitely many $G$-translates of $\D$ cover $\X$.
Dually, we say that $\D$ is \emph{thick\/} under the action of $G$ or \emph{$G$-thick\/} for short, if the intersection of any finitely many $G$-translates of $\D$ is consistent (in $\X$).

The same properties may be attributed to formulas (as these are identified with the set they define).
When these properties are attributed to a type $p(x)$, we understand that they hold for every conjunction of formulas in $p(x)$.

Our terminology is taken from topological dynamics.
In similars context syndetic sets are called \textit{generic,} see eg.~\cite{Ne}.
In~\cite{CK} the authors write \textit{quasi-non-dividing\/} for \textit{thick\/} under the action of ${\rm Aut}(\U/A)$.

Notation: for $\D\subseteq\X$ and $C\subseteq G$ we write \emph{$C{\cdot}\D$\/} for $\{h{\cdot}\D: h\in C\}$.

In this chapter many proofs require some juggling with negations as epitomized by the following fact which is proved by spelling out the definitions

\begin{fact}\label{fact_fip}
  The following are equivalent
  \begin{itemize}
    \item[1.] $\D$ is not $G$-syndetic
    \item[2.] $\neg\D$ is $G$-thick.
  \end{itemize}\smallskip
\end{fact}

% \begin{proof}
%   Spelling out the definitions, (1) and (2) are, respectively, equivalent to
%   \begin{itemize}
%     \item[1$'$.] there are no finite $C\subseteq G$ such that $\Y\ \subseteq\ \cup\, C{\cdot}\D$.
%     \item[2$'$.] $\varnothing\ \neq\ \Y\,\cap\,\big(\cap\, C{\cdot}\neg\D\big)$ for every finite $C\subseteq G$.
%   \end{itemize} 
%   Then the equivalence is evident.
% \end{proof}


% The following characterization is also useful (an is sometimes taken as the definition).

% \begin{corollary}\label{fact_fip2}
%   Work relative to some given $\Y\subseteq\X$.
%   The following are equivalent
%   \begin{itemize}
%     \item[1.] $\D$ is $G$-thick
%     \item[2.] $\D\cap\C\neq\varnothing$ for every set $\C$ that is $G$-syndetic.
%   \end{itemize}
% \end{corollary}

% \begin{proof}
%   1$\Rightarrow$2. 
%   If $\D$ is $G$-thick and $\D\cap\C=\varnothing$ then $\D\subseteq\neg\C$.
%   Then $\neg\C$ is also $G$-thick which, by Fact~\ref{fact_fip} implies that $\C$ is not $G$-syndetic.

%   2$\Rightarrow$1. 
%   If $\D$ is not $G$-thick then $\D$ is $G$-syndetic then $\neg$2.
% \end{proof}

Define the following type

\ceq{\hfill\emph{$\Sigma_G(x)$}}{=}{\big\{\vartheta(x)\in \BDelta({\Z})\ :\ \vartheta(x)\textrm{ is }G\textrm{-syndetic}\big\}.}

% \begin{corollary}\label{corol_q_pers}
%   Work relative to some given $\Y\subseteq\X$.
%   Let $\D\in\BDelta(\Z)$ and assume that $\Sigma_{G,\Y}(x)\cup\{x\in\D\}$ is finitely consistent in $\Y$.
%   Then $\D$ is $G$-thick.
% \end{corollary}

% \begin{proof}
%   As $\Sigma_{G,\Y}(x)\cup\{x\in\D\}$ is finitely consistent in $\Y$, it cannot be that $\neg\D$ is $G$-syndetic.
%   Then from Fact~\ref{fact_fip} we obtain that $\D$ is $G$-thick.
% \end{proof}

\begin{theorem}\label{thm_syndetic_invariant}
  For every $p(x)\in S_\Delta(\Z)$ the following are equivalent
  \begin{itemize}
    \item[1.] $p(x)$ is $G$-invariant
    \item[2.] $p(x)\vdash\Sigma_G(x)$
    \item[3.] $p(x)$ is $G$-thick.
  \end{itemize}
\end{theorem}

\begin{proof}
  1$\Rightarrow$2.
  Let $\D\in\BDelta(\Z)$ be a $G$-syndetic.
  Pick $C\subseteq G$ be finite such that $\X\subseteq\cup\,C{\cdot}\D$.
  Then $p(x)$ is finitely consistent with $x\in\cup\,C{\cdot}\D$.
  By completeness, $p(x)\vdash x\in h\,\D$ for some $h\in C$.
  Finally, by invariance, $p(x)\vdash x\in\D$.
  
  2$\Rightarrow$3.
  Let $\varphi(x)\in p$.
  As $\Sigma_G(x)\cup\{\varphi(x)\}$ is finitely consistent, it cannot be that $\neg\varphi(x)$ is $G$-syndetic.
  Then from Fact~\ref{fact_fip} we obtain that $\varphi(x)$ is $G$-thick.

  3$\Rightarrow$1.
  Negate 1.
  Let $\D\in\BDelta(\Z)$ and $g\in G$ such that $p(x)\vdash x\in\D$ and $p(x)\nvdash g{\cdot}\D$.
  By completeness $p(x)\vdash x\in(\D\cap\neg g{\cdot}\D)$.
  Clearly $\D\cap \neg g{\cdot}\D$ is not $G$-thick as it is inconsistent with its $g$-translate.
\end{proof}


The following theorem gives a necessary and sufficient condition for the  existence of global $G$-invariant $\BDelta(\Z)$-type.
Ideally, we would like that every $G$-thick $\BDelta(Z)$-type extends to a global thick type.
Unfortunately this is not true in general (it is an assumption with important consequences, see Section~\ref{tame_landscape}).

A set $\D$ is \emph{$G$-wide\/} relative to $\Y$ if every finite cover of $\D\cap\Y$ by $\BDelta(\Z)$-sets contains a set that is $G$-thick relative to $\Y$.
A type is $G$-wide if every conjunction of formulas in the type is $G$-wide.

In~\cite{CK} a similar property is called \textit{quasi-non-forking.}
Our use of the term \textit{wide\/} is consistent with~\cite{Hr}, though we apply it to a narrow context.

\begin{theorem}\label{thm_syndetic_invariant2}
  For every $\D\in\BDelta(\Z)$ the following are equivalent 
  \begin{itemize}
    \item[1.] $\Sigma_G(x)\cup\{x\in\D\}$ is finitely consistent
    \item[2.] there is a $G$-thick type $p(x)\in S_\Delta(\Z)$ that entails $x\in\D$
    \item[3.] $\D$ is $G$-wide.\smallskip
  \end{itemize}
\end{theorem}

\begin{proof}
  1$\Rightarrow$2.
  By Theorem~\ref{thm_syndetic_invariant}, it suffices to pick any $p(x)\in S_\Delta(\Z)$ extending the type $\Sigma_G(x)\cup\{x\in\D\}$.

  2$\Rightarrow$1.
  By Theorem~\ref{thm_syndetic_invariant}.

  2$\Rightarrow$3.
  Let $\C_1,\dots,\C_n$ be $\BDelta({\Z})$-sets that cover $\D$.
  Pick $p(x)$ as in 2.
  By completeness, $p(x)\vdash x\in\C_i$ for some $i$.
  Therefore, $\C_i$ is $G$-thick.

  3$\Rightarrow$2.
  Let $p(x)$ be maximal among the $\BDelta(\Z)$-types that are finitely consistent with $\D$ and are $G$-wide.
  We claim that $p(x)$ is a complete $\BDelta(\Z)$-type.
  Suppose for a contradiction that $\vartheta(x),\neg\vartheta(x)\notin p$.
  By maximality there is some formula $\psi(x)$, a conjunction of formulas in $p(x)$, and some $\BDelta({\Z})$-sets $\C_1,\dots,\C_n$ that cover both $\psi(\X)\cap\vartheta(\X)$ and $\psi(\X)\smallsetminus\vartheta(\X)$ and such that no $\C_i$ is $G$-thick.
  As $\C_1,\dots,\C_n$ cover $\psi(\X)$ this is a contradiction.
  It is only left to show that $p(x)$ is $G$-thick.
  This follows from completeness and Theorem~\ref{thm_syndetic_invariant}.
\end{proof}


%%%%%%%%%%%%%%%%%%%%%%%%%%
%%%%%%%%%%%%%%%%%%%%%%%%%%
%%%%%%%%%%%%%%%%%%%%%%%%%%
%%%%%%%%%%%%%%%%%%%%%%%%%%
%%%%%%%%%%%%%%%%%%%%%%%%%%
\section{Connected components}\label{G0}
\def\medrel#1{\parbox[t]{5ex}{$\displaystyle\hfil #1$}}
\def\ceq#1#2#3{\parbox[t]{12ex}{$\displaystyle #1$}\medrel{#2}{$\displaystyle #3$}}

Let \emph{$G$\/} $\le{\rm Aut}(\UDelta)$.
Unfortunately, syndeticity is not preserved under intersection.
In particular $\Sigma_G(x)$ is not a $G$-syndetic type, and it may even be inconsistent.
Then following notion is relevant.

\begin{definition}\label{def_Q}\ 

  \ceq{\hfill\emph{$Q_G$\/}}{=}{\big\{q(x)\subseteq\Sigma_G(x)\;:\ q(x)\textrm{ maximally }G\textrm{-syndetic}\big\}.}\smallskip

  In other words, the types in $Q_G$ are maximal among the subtypes of $\Sigma_G(x)$ that are closed under conjunction.
\end{definition}

It is easy to see that $Q_G$ is closed under the action of $G$.
We write ${\rm Stab}(q)$ for the stabilizer of $q(x)\subseteq\BDelta(\Z)$ in $G$, that is, the subgroup $\{g\in G\ :\ g{\cdot}q(x)=q(x)\}$.
We write ${\rm Stab}(\D)$ with a similar meaning.
Finally we define

\ceq{\hfill\emph{$G^1$}}{=}{{\rm Stab}(Q_G)\medrel{=}\bigcap_{q\in Q_G}{\rm Stab}(q).}

It is easy to verify that $G^1\trianglelefteq G$.

\begin{proposition}\label{prop_StabQ}\ \smallskip

  \ceq{\hfill G^1}{=}{\big\{g\in G\ :\ \D\cap g{\cdot}\D\in\Sigma_G(x)\textrm{ whenever }\D\in\Sigma_G(x)\big\}.}\smallskip
\end{proposition}

\begin{proof}
  $\subseteq$.
  Pick any $k\in G^1$ and $\D\in\Sigma_G(x)$.
  Let $q(x)\in Q_G$ be a type containing $x\in\D$.
  From the $G^1$-invariance of $q(x)$ we obtain that $q(x)\vdash x\in k{\cdot}\D$.
  Then $q(x)\vdash x\in \D\cap k{\cdot}\D$, hence $\D\cap k{\cdot}\D$ is $G$-syndetic.

  $\supseteq$.
  Pick any $g\notin G^1$.
  Then $q(x)\neq g{\cdot}q(x)$ for some $q(x)\in Q_G$.
  Let $\varphi(x)\in q$ such that $q(x)\not\vdash g{\cdot}\varphi(x)$.
  By maximality, $\psi(x)\wedge g{\cdot}\varphi(x)$ is not $G$-syndetic for some $\psi(x)\in q$.
  As $q(x)$ is closed under conjunction, we can assume $\varphi(x)=\psi(x)$, then $g$ does not belong to the set on the r.h.s.
\end{proof}


%%%%%%%%%%%%%%%
\begin{definition}\label{def_P}\ 

  \ceq{\hfill\emph{$P_G$\/}}{=}{\big\{p(x)\subseteq\Sigma_G(x)\;:\ p(x)\textrm{ maximally finitely consistent}\big\}.}\smallskip

  
\end{definition}

Along the same lines as above, we define

\ceq{\hfill\emph{$G^{1\!1}$}}{=}{{\rm Stab}(P_G)\medrel{=}\bigcap_{p\in P_G}{\rm Stab}(p).}

% It is easy to verify that $G^{11}\trianglelefteq G$.
By the following proposition, $G^1\le G^{1\!1}\trianglelefteq G$.
We do not know if the inclusion is strict.

\begin{proposition}\label{prop_StabP}\ 

  \ceq{\hfill G^{1\!1}}{=}{\big\{g\in G\ :\ q(x)\cup g{\cdot}q(x)\textrm{ is finitely consistent for every }q(x)\in Q_G\big\}.}\smallskip
\end{proposition}

\begin{proof}
  $\subseteq$.
  Pick any $k\in G^{1\!1}$ and $q(x)\in Q_G$.
  Let $p(x)\in P_G$ be a type containing $q(x)$.
  From the $G^{1\!1}$-invariance of $p(x)$ we obtain that $p(x)\vdash k{\cdot}q(x)$.
  Then $q(x)\cup k{\cdot}q(x)$ is finitely consistent.

  $\supseteq$.
  Pick any $g\notin G^{1\!1}$.
  Then $q(x)\cup g{\cdot}q(x)$ is not finitely consistent for some $q(x)\in Q_G$.
  Let $\varphi(x)\in q$ such that $\varphi(x)\wedge g{\cdot}\varphi(x)$ is inconsistent.
  But $\varphi(x)\in p$ then $g$ does not belong to the set on the r.h.s.
\end{proof}

\begin{theorem}\label{thm_gammaK}
  Any finite conjunction of formulas in $\Sigma_{G^1}(x)$ is $G$-syndetic.
  In particular $\Sigma_{G^1}(x)$ is finitely consistent.
\end{theorem}

\begin{proof}
  \def\medrel#1{\parbox[t]{5ex}{$\displaystyle\hfil #1$}}
  \def\ceq#1#2#3{\parbox[t]{23ex}{$\displaystyle #1$}\medrel{#2}{$\displaystyle #3$}}
  Notice that from Proposition~\ref{prop_StabQ} it easily follows that for every $\D\in\Sigma_G$ and every finite $F\subseteq G^1$ the set $\cap\,F{\cdot}\D$ is $G$-syndetic.
   
  Let $\D_1,\dots,\D_n\in\Sigma_{G^1}$.
  Assume inductively that $\D_1\cap\dots\cap\D_{n-1}$ is $G$-syndetic.
  Let $F\subseteq G^1$ be such that $\cup\,F{\cdot}\D_n=\X$.
  Then
  
  \ceq{1.\hfill\cup\,F{\cdot}[\D_1\cap\dots\cap\D_n]}
  {\supseteq}{\cap\, F{\cdot}[\D_1\cap\dots\cap\D_{n-1}]\ \cap\ \cup\,F{\cdot}\D_n}

  \ceq{}{=}{\cap\, F{\cdot}[\D_1\cap\dots\cap\D_{n-1}].}

  This last set is $G$-syndetic by the inductive hypothesis and what remarked above.
  The $G$-syndeticity of $\D_1\cap\dots\cap\D_n$ follows.
\end{proof}

Unfortunately, we are unable to conclude that the intersection of $G^1$-syndetic sets is $G^1$-syndetic.

A similar argument shows the following.

\begin{theorem}\label{thm_gammaP}
   $\Sigma_{G^{1\!1}}(x)$ is finitely consistent.
\end{theorem}

\begin{proof}
  \def\medrel#1{\parbox[t]{5ex}{$\displaystyle\hfil #1$}}
  \def\ceq#1#2#3{\parbox[t]{23ex}{$\displaystyle #1$}\medrel{#2}{$\displaystyle #3$}}   
  Let $\D_1,\dots,\D_n\in\Sigma_{G^{1\!1}}$.
  Assume inductively that $\D_1\cap\dots\cap\D_{n-1}\neq\varnothing$.
  As the $\D_i$ are in particular $G$-syndetic, some $p(x)\in P_G$ entails this disjunction.
  Let $F\subseteq G^{1\!1}$ be such that $\cup\,F{\cdot}\D_n=\X$.
  By $G^{1\!1}$-invariance, $p(x)$ also entails $\cap\, F{\cdot}[\D_1\cap\dots\cap\D_{n-1}]$.
  Then by (1) in the above proof $\,\cup\,F{\cdot}[\D_1\cap\dots\cap\D_n]\neq\varnothing$ .
  The consistency of $\D_1\cap\dots\cap\D_n$ follows.
\end{proof}
%$$$$$$$$$$$$$

By Theorems~\ref{thm_syndetic_invariant2} from the above theorems it follows that $G^1$-syndetic sets are $G^1$-wide, similarly for $G^{1\!1}$.
But we can do better.
First, we remark a useful consequence of normality.

\begin{remark}\label{rem_invariance_normalsubg}
\def\medrel#1{\parbox[t]{5ex}{$\displaystyle\hfil #1$}}
\def\ceq#1#2#3{\parbox[t]{20ex}{$\displaystyle #1$}\medrel{#2}{$\displaystyle #3$}}
  Assume $H\trianglelefteq G$.
  For every $\D\subseteq\X$ and every $g\in G$ \smallskip
  
  \ceq{\hfill\D\textrm{ is }H\textrm{-foo}}{\Rightarrow}{g{\cdot}\D\textrm{ is }H\textrm{-foo},} \smallskip
  
  where \textit{foo\/} can be replaced by \textit{syndetic,} \textit{invariant,} \textit{thick,} \textit{wide.}
  In particular, the type $\Sigma_H(x)$ is $G$-invariant.
\end{remark}

Recall that when $\Sigma_H(x)$ is finitely consistent then $H$-syndetic sets are $H$-wide, see Theorem~\ref{thm_syndetic_invariant2}.
As it happens, under the assumption of normality, this can be strengthened as follows.

\begin{proposition}\label{prop_Gsyndetic_Hthick1}
  Assume $H\trianglelefteq G$ and that $\Sigma_H(x)$ is finitely consistent.
  Then every $G$-syn\-detic $\D\in\BDelta(\Z)$ is $H$-wide.
  In particular all types in $Q_G$ are $G^{1\!1}$-wide.
\end{proposition}

\begin{proof}
  Let $p(x)\in S_\Delta(\Z)$ be finitely consistent with $\Sigma_H(x)$.
  As $\D$ is $G$-syndetic, by completeness $p(x)\vdash x\in g{\cdot}\D$ for some $g\in G$.
  Equivalently, $g^{-1}\kern-.5ex\cdot p(x)\vdash x\in\D$.
  As $p(x)$ is $H$-thick, by Remark~\ref{rem_invariance_normalsubg} also $g^{-1}\kern-.5ex\cdot p(x)$ is $H$-thick.
  Then the proposition follows from Theorem~\ref{thm_syndetic_invariant2}.
\end{proof}

When the context suggests it, $\BDelta(\Z)$ will denote the broader class that includes also formulas obtained substituting $y$ for $x$ and is closed under Boolean connectives.
A $\BDelta(\Z)$-definable equivalence relation is an equivalence relation defined by some formula $\varepsilon(x,y)\in\BDelta(\Z)$.
We define $\BDelta(\Z)$-type-definable equivalence relations similarly.
Note that if $\varepsilon(x,y)$ is a $\BDelta(\Z)$-definable, or type-definable, equivalence relation then $\varepsilon(x,a)$, for any $a\in\X$, is equivalent to a $\BDelta(\Z)$-formula, respectively a $\BDelta(\Z)$-type.

An equivalence relation is bounded if it has a small number of equivalence classes.

\begin{definition}\label{def_G00}
  Let $\Phi^0$ be the collection of sets that are equivalence classes of a bounded $G$-invariant $\BDelta(\Z)$-type-definable equivalence relation.
  Let $\Phi^{00}$ be the collection of sets that are equivalence classes of a bounded $G$-invariant $\BDelta(\Z)$-type-definable equivalence relation.
  Define
  \smallskip

  \ceq{\hfill\emph{$G^0$}}{=}{{\rm Stab}(\Phi^0)}\medrel{=}$\displaystyle\bigcap\big\{{\rm Stab}(\D)\ :\  \D\in\Phi^0\}$

  and

  \ceq{\hfill\emph{$G^{00}$}}{=}{{\rm Stab}(\Phi^{00})}\medrel{=}$\displaystyle\bigcap\big\{{\rm Stab}(\Y)\ :\  \Y\in\Phi^{00}\}$.

\end{definition}

\begin{proposition}
  Let $\X$ be type-definable.
  Assume that $G$ acts transitively on $\X$.
  Then 
  
  \ceq{\hfill G^{1\!1}}{\leq}{G^{00}}\medrel{\leq}$G^0$.
\end{proposition}

\begin{proof}
  The inclusion $G^{00}\leq G^0$ is trivial.
  Let $\Y\in\Phi^{00}$ and $\Y\subseteq\D\in\BDelta(\Z)$.
  Then a small number of $G$-translates of $\D$ cover $\X$.
  Therefore, by compactness, $\D$ is $G$-syndetic.
  Then the type defining $\Y$ is $G$-syndetic.
  Then this type extends to some $p(x)\in P_G$.
  By $G^{1\!1}$-in\-variance $p(x)$ entails also $k{\cdot}\Y$ for every $k\in G^{1\!1}$.
  Then $k$ is in the stabilizer of $\Y$.
\end{proof}



%%%%%%%%%%%%%%%%%%%%%%%%%%%
%%%%%%%%%%%%%%%%%%%%%%%%%%%
%%%%%%%%%%%%%%%%%%%%%%%%%%%
%%%%%%%%%%%%%%%%%%%%%%%%%%%
%%%%%%%%%%%%%%%%%%%%%%%%%%%
\section{Strong syndeticity}\label{strong_syndeticity}
\def\medrel#1{\parbox[t]{5ex}{$\displaystyle\hfil #1$}}
\def\ceq#1#2#3{\parbox[t]{20ex}{$\displaystyle #1$}\medrel{#2}{$\displaystyle #3$}}

% Unfortunately, syndeticity is not preserved under intersection.
% To obtain closure under intersection, we need to push the concept to a higher level of complexity.

Work relative to some given $\Y\subseteq\X$.
A set $\D$ is \emph{strongly $G$-syndetic\/} if for every finite $F\subseteq G$ the set $\cap\,F{\cdot}\D$ is $G$-syndetic (recall that $F{\cdot}\D$ stands for $\{h{\cdot}\D : h\in F\}$).
Dually, we say that $\D$ is \emph{weakly $G$-thick\/} if for some finite $F\subseteq G$ the set $\cup\,F{\cdot}\D$ is thick.
Again, the same properties may be attributed to formulas and types when every conjunction of formulas in the type has the property.

\noindent\llap{\textcolor{red}{\Large\warning}\kern1.5ex}\ignorespaces
In topological dynamic, strong syndedic sets are called \textit{thickly syndetic\/} and weak thickness is called \textit{piecewise syndetic.}
Newelski in~\cite{Ne} says \textit{weak generic\/} for weakly thick.
These are terminologies that defy my intuition.

\begin{lemma}\label{lem_strongly_syndetic}
  Work relative to some given $\Y\subseteq\X$.
  The intersection of two strongly $G$-syndetic sets is strongly $G$-syndetic.
\end{lemma}

\begin{proof}
  Let $\D$ and $\C$ be strongly $G$-syndetic and let $C\subseteq G$ be an arbitrary finite set.
  It suffices to prove that $\B=\cap\, C{\cdot}(\C\cap\D)$ is $G$-syndetic. 
  Clearly $\B=\C'\cap\D'$, where $\C'=\cap\, C{\cdot}\C$ and $\D'=\cap\, C{\cdot}\D$.
  Note that $\C'$ and $\D'$ are both strongly $G$-syndetic.
  In particular $\Y\subseteq\cup\,F{\cdot}\D'$ for some finite $F\subseteq G$.
  Note that
  
  \ceq{\hfill\cup\,F{\cdot}\B}{=}{\cup\,F{\cdot}\big[\C'\ \cap\ \D'\big]}
  
  \ceq{}{\supseteq}{ \big(\cap\, F{\cdot}\C'\big)\ \cap\ \big(\cup\,F{\cdot}\D'\big)}
  
  \ceq{}{=}{\cap\, F{\cdot}\C'}
  
  As $\C'$ is strongly $G$-syndetic, $\cap\, F{\cdot}\C'$ is $G$-syndetic.
  Therefore $\cup\,F{\cdot}\B$ is also $G$-syndetic.
  The $G$-syndeticity of $\B$ follows.
\end{proof}

Define the following type

\ceq{\hfill\emph{${\mathstrut}^{\rm s}\kern-.1ex\Sigma_{G,\Y}(x)$}}{=}{\big\{\vartheta(x)\in \BDelta({\Z})\ :\ \vartheta(x)\textrm{ is strongly }G\textrm{-syndetic relative to }\Y\big\}.}\smallskip

Note that Theorem~\ref{thm_gammaK} shows that $\Sigma_{G^1}(x)\subseteq{\mathstrut}^{\rm s}\kern-.2ex\Sigma_G(x)$.

\begin{corollary}\label{corol_str_gen}
  Work relative to some given $\Y\subseteq\X$.
  Then ${\mathstrut}^{\rm s}\kern-.1ex\Sigma_{G,\Y}(x)$ is finitely consistent, strongly $G$-syndetic, and $G$-invariant.\smallskip

  Moreover, ${\mathstrut}^{\rm s}\kern-.1ex\Sigma_{G,\Y}(x)$ is the maximal $G$-syndetic $G$-invariant type and\medskip


  \ceq{\hfill{\mathstrut}^{\rm s}\kern-.1ex\Sigma_{G,\Y}(x)}{=}{\bigcap_{q\in Q_{G,\Y}}q(x)}.\smallskip

  Where $Q_{G,\Y}$ is as in Definition~\ref{def_Q} but relativized to $\Y$.  
\end{corollary}

\begin{proof}
  Strong $G$-syndeticity is an immediate consequence of Lemma~\ref{lem_strongly_syndetic}.
  Finite consistency is a consequence of syndeticity.
  Finally, $G$-invariance is clear because any translate of a strongly $G$-syndetic formula is also strongly $G$-syndetic.
\end{proof}

% The following is similar to Corollary~\ref{corol_q_pers}, but now we obtain also the converse implication.

\begin{corollary}\label{corol_q_w_pers}
  Work relative to some given $\Y\subseteq\X$.
  For every $\D\in\BDelta(\Z)$ the following are equivalent
  \begin{itemize}
    \item [1.] ${\mathstrut}^{\rm s}\kern-.1ex\Sigma_{G,\Y}(x)\cup\{x\in\D\}$ is finitely consistent
    \item [2.] $\D$ is weakly $G$-thick.
  \end{itemize}
\end{corollary}

\begin{proof}
  1$\Rightarrow$2. 
  If ${\mathstrut}^{\rm s}\kern-.1ex\Sigma_{G,\Y}(x)\cup\{x\in\D\}$ is finitely consistent, then $\neg\D$ is strongly $G$-syndetic.
  From Fact~\ref{fact_fip}, we obtain that $\neg\D$ not being strongly $G$-syndetic is equivalent to $\D$ being weakly $G$-thick.

  2$\Rightarrow$1.
  Suppose ${\mathstrut}^{\rm s}\kern-.1ex\Sigma_{G,\Y}(x)\vdash x\notin\D$.
  Then $\neg\D$ is strongly $G$-syndetic.
  From Fact~\ref{fact_fip}, $\D$ is not weakly $G$-thick.
\end{proof}

The following theorem asserts that weak thickness is partition regular.

\begin{theorem}\label{thm_wt_partreg}
  Work relative to some given $\Y\subseteq\X$.
  If $\C\cup\B$ is weakly $G$-thick then $\B$ or $\C$ is weakly $G$-thick.
\end{theorem}

\begin{proof}
  As ${\mathstrut}^{\rm s}\kern-.1ex\Sigma_{G,\Y}(x)$ is closed under conjunction.
  If $x\in \C\cup\B$ is finitely consistent with ${\mathstrut}^{\rm s}\kern-.1ex\Sigma_{G,\Y}(x)$ then so is one of the two sets.
\end{proof}

%%%%%%%%%%%%%%%%%%%%%%%%%%%%%%%%%%%
%%%%%%%%%%%%%%%%%%%%%%%%%%%%%%%%%%%
%%%%%%%%%%%%%%%%%%%%%%%%%%%%%%%%%%%
%%%%%%%%%%%%%%%%%%%%%%%%%%%%%%%%%%%
%%%%%%%%%%%%%%%%%%%%%%%%%%%%%%%%%%%
\section{A tamer landscape}\label{tame_landscape}

Under suitable assumptions some notions introduced in this chapter coalesce, and we are left with a tamer landscape.

\begin{theorem}\label{thm_coalesce}
  Work relative to some given $\Y\subseteq\X$.
  The following are equivalent
  \begin{itemize}
    \item[1.] $H$-thick $\BDelta({\Z})$-sets are $H$-wide
    \item[2.] $H$-syndetic $\BDelta({\Z})$-sets are closed under intersection 
    \item[3.] $H$-syndetic $\BDelta({\Z})$-sets are strongly $H$-syndetic
    \item[4.] weakly $H$-thick $\BDelta({\Z})$-sets are $H$-thick.
  \end{itemize}
\end{theorem}

\begin{proof}
  Clearly 2$\Leftrightarrow$3$\Leftrightarrow$4.

  1$\Rightarrow$2.
  Let $\C$ and $\D$ be $H$-syndetic $\BDelta({\Z})$-sets.
  Suppose for a contradiction that $\C\cap\D$ is not $H$-syndetic.
  Then $\neg(\C\cap\D)$ is $H$-thick.
  By 1 and Theorem~\ref{thm_syndetic_invariant2} there is a $H$-invariant type $p(x)\in S_\Delta(\Z)$ that is finitely consistent in $\Y$ and such that $p(x)\vdash x\notin\C\cap\D$.
  By completeness either $p(x)\vdash x\notin\C$ or $p(x)\vdash x\notin\D$.
  This is a contradiction because by Theorem~\ref{thm_syndetic_invariant} $p(x)\vdash x\in\C$ and $p(x)\vdash x\in\D$.

  % 2$\Rightarrow$1.
  % By 2, $\Sigma_H(x)$ is consistent.
  % Then $\C$ is $H$-syndetic if and only if $\Sigma_H(x)\vdashx\in\C$.
  % If $\D$ is $H$-thick, then $\neg\D$ is not $H$-syndetic.
  % Then $\Sigma_H(x)\notprovesx\notin\D$ by what remarked above.
  % Then $\D$ is $H$-wide by Theorem~\ref{thm_syndetic_invariant2}.

  4$\Rightarrow$1. By Theorem~\ref{thm_wt_partreg}
\end{proof}

% It is easy to see that the equivalent conditions in Theorem~\ref{thm_coalesce} hold when $H=G^\omega$, as defined in Section~\ref{G0}, and $\Y=\X$.
% Vice vesa, if the equivalent conditions in Theorem~\ref{thm_coalesce} hold for $H\le G$, then $H\le G^\omega$. 

A subset $\Y\subseteq\X$ is \emph{$H$-stationary\/} if there exists a unique $H$-invariant $p(x)\in S_{\Delta,\Y}(\Z)$.
When $\Sigma_{H,\Y}(x)$ is consistent, by 1 in Theorem~\ref{thm_syndetic_invariant2} this is equivalent to requiring that $\D$ and $\neg\D$ are not both wide relative to $\Y$.
The conditions in Theorem~\ref{thm_coalesce} together with stationarity, produce further simplification.

\begin{fact}\label{fact_starionarity_thick_syndetic}
  Work relative to some given $\Y\subseteq\X$.
  Assume the equivalent conditions in Theorem~\ref{thm_coalesce} hold.
  Suppose that the set $\Y$ is $H$-stationary.
  Then the following are equivalent
  \begin{itemize}
    \item [1.] $\D$ is $H$-syndetic
    \item [2.] $\D$ is $H$-wide.
  \end{itemize}\smallskip
\end{fact}

\begin{proof}
  1$\Rightarrow$2.
  By assumption, $\Sigma_{H,\Y}(x)$ is consistent.
  Then $H$-syndetic $\BDelta({\Z})$-sets are $H$-thick by Corollary~\ref{corol_def_mu}.
  Finally, they are  $H$-wide by 1 in Theorem~\ref{thm_coalesce}.
  (Stationarity is not required in this direction.)

  2$\Rightarrow$1.
  If $\D$ was not $H$-syndetic then $\neg\D$ would be $H$-thick and therefore, by the assumption, $H$-wide.
  This contradicts $H$-stationarity.
\end{proof}

