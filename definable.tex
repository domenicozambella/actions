% !TEX root = actions.tex
\section{Definable actions}
\label{definable}


\def\ceq#1#2#3{\noindent\parbox[t]{15ex}{$\displaystyle #1$}\parbox{6ex}{\hfil $#2$}{$\displaystyle #3$}}

In this section $\Z$ is a type-definable group that acts on a type definable set $\X$.
The group operations and the group action are assumed to be definable.
We use the symbol $\,\cdot\,$ for both the group multiplication and the group action.
Let $M\preceq\U$.
In this section $\Delta$ contains formulas $\varphi(z\,;x)$ of the form $\psi(z^{-1}\!\cdot x)$ for some $\psi(x)\in L(M)$.
Clearly, the sets $\varphi(\Z\,;\X)$ are $\Z$-invariant.
We write $1$ for the identity of $\Z$.
%Then $\LDelta_x(\{1\})$ coincide up to equivalence with $L_x(M)$.

It is worth noticing that automorphisms of $\UDelta$ need not preserve the group operations nor the group action.

To each $h\in\Z$ we associate the $\LDelta$-automorphism $\langle g\,;a\rangle\mapsto\langle h{\cdot}g\,;h{\cdot}a\rangle$.
Therefore $\Z$ is, up to isomorphism, a subgroup of ${\rm Aut}(\UDelta)$.
Note that $g\cdot\varphi(h\,;\X)=\varphi(g{\cdot}h\,;\X)$ for all $g,h\in\Z$ and $\varphi(z\,;x)\in\Delta$.
Therefore, the orbit of $\varphi(\Z\,;\X)$ under the action of $\Z$ is $\{\varphi(g\,;\X) : g\in\Z\}$.
It follows that the orbit of $\varphi(h\,;\X)$ under the actions of ${\rm Aut}(\UDelta)$ coincides with the orbit under the action of $\Z$.

We write $X$ and $Z$ for $\X\cap M^x$ and $\Z\cap M^z$ respectively.
Clearly $Z\le{\rm Aut}(\UDelta/\{Z\})$, the latter is the subgroup of ${\rm Aut}(\UDelta)$ that fixes $Z$ setwise.
By the observation above, when $h\in Z$ and $\varphi(z\,;x)\in\Delta$, the orbit of $\varphi(h\,;x)$ under the action of ${\rm Aut}(\UDelta/\{Z\})$ coincides with the orbit under the action of $Z$.
Also note that, by how $\Delta$ is defined in this section, up to equivalence we $\Delta(\{1\})$, $\Delta(Z)$, and $\BDelta(Z)$ coincide with $L_x(M)$.

When $g\in\Z$ and $a\in\X$ we write $g\nonfork a$ if for every $\varphi(z\,;x)\in\Delta$ such that $\varphi(g\,;a)$ we have that $\varphi(Z\,;a)\neq\varnothing$.
When $\C\subseteq\Z$ and $\D\subseteq\X$ we define

\ceq{\hfill\C\circ\D}{=}{\Big\{g{\cdot}a\ :\ g\in\C,\ a\in\D,\ g\nonfork a\Big\}}

If $G\le{\rm Aut}(\UDelta)$ and $a\in\X$, we write $G{\cdot}a$ for the orbit of $a$ under the action of $G$.
We abbreviate $\C\circ\big({\rm Aut}(\UDelta/M)\cdot a\big)$ with $\C\circ a$.

If $\A\subseteq\X$ we write ${\rm cl}(\A)$ for the closure of $\A$ in the topology generated by the subsets of $\X$ that are $L(M)$-definable.
Explicitely,

\ceq{1.\hfill{\rm cl}(\A)}{=}{\bigcap\Big\{\varphi(\X)\ :\ \varphi(x)\in\LDelta(M)\text{ such that }\A\subseteq\varphi(\X)\Big\}}

\ceq{2.}{=}{\Big\{b\ :\ \varphi(\A)\neq\varnothing\text{ for every }\varphi(x)\in L(M)\text{ that is satisfied by }b\Big\}}


% \begin{fact}
%   For every $\A\subseteq\X$\smallskip

%   \ceq{\hfill{\rm cl}(\cup\,Z{\cdot}\A)}{=}{\Z\circ\A}
% \end{fact}

% \begin{proof}
%   $\supseteq$. 
%   Let $b\in\Z\circ\A$, say $b=g{\cdot}a$ for some $g\in\Z$ and $a\in\A$ such that $g\nonfork a$.
%   Let $\varphi(x)\in\LDelta(M)$ be such that $Z{\cdot}\A\subseteq\varphi(\X)$.
%   Then $g\nonfork a$ implies $\varphi(g{\cdot}a)$.
%   This proves that $b\in{\rm cl}(\cup\,Z{\cdot}\A)$.
  
%   $\subseteq$. 
%   Let $b\in{\rm cl}(\cup\,Z{\cdot}\A)$ be given and define $p(x)=\LDelta\mbox{-tp}(b/M)$.
%   It suffices to prove that for every $a\in\A$ the type $z\nonfork a$ is consistent with $p(z{\cdot}a)$.
%   In fact, if $g\nonfork a$, then $g{\cdot}a\in\Z\circ a$.
%   And, as $\Z\circ a$ is $M$-invariant, if $p(g{\cdot}a)$ then $b\in\Z\circ a$.
%   Now, to prove concistency, assume for a contradiction that $z\nonfork a\rightarrow\neg\varphi(z{\cdot}a)$ for some $\varphi(x)\in p$.
%   As any $m\in Z$ realizes $z\nonfork a$, no element of $Z{\cdot}a$ satisfies $\varphi(x)$.
%   This is a contradiction by (2) above.
% \end{proof}


\begin{fact}
  For every $a\in\X$\smallskip

  \ceq{\hfill{\rm cl}(Z{\cdot}a)}{=}{\Z\circ a}
\end{fact}

\begin{proof}
  $\supseteq$. 
  Let $b\in\Z\circ a$, say $b=g{\cdot}a$ for some $g\nonfork a$.
  Let $\varphi(x)\in\LDelta(M)$ be such that $Z{\cdot}a\subseteq\varphi(\X)$.
  Then $g\nonfork a$ implies $\varphi(g{\cdot}a)$.
  This proves that $b\in{\rm cl}(Z{\cdot}a)$.
  
  $\subseteq$. 
  Let $b\in{\rm cl}(Z{\cdot}a)$ be given and define $p(x)=\LDelta\mbox{-tp}(b/M)$.
  It suffices to prove that $z\nonfork a$ is consistent with $p(z{\cdot}a)$.
  In fact, if $g\nonfork a$, then $g{\cdot}a\in\Z\circ a$.
  And, as $\Z\circ a$ is $M$-invariant, if $p(g{\cdot}a)$ then $b\in\Z\circ a$.
  Now, to prove concistency, assume for a contradiction that $z\nonfork a\rightarrow\neg\varphi(z{\cdot}a)$ for some $\varphi(x)\in p$.
  As any $m\in Z$ realizes $z\nonfork a$, no element of $Z{\cdot}a$ satisfies $\varphi(x)$.
  This is a contradiction by (2) above.
\end{proof}

% A subset $A\subseteq\X$ is dense if $\psi(A)\neq\varnothing$ for every $\psi(x)\in\Psi$.

Let $\D\subseteq\X$ be an $M$-definable set.
As $Z$ is small, $\D$ is $Z$-thick if and only if $\cap\,Z{\cdot}\D$ is consistent.

\begin{fact}\label{fact_thick_circ}
  Let $\D\subseteq\X$ be definable over $M$.
  Then the following are equivalent
  \begin{itemize}
    \item [1.] $\D$ is $Z$-thick
    \item [2.] $\Z\circ a\subseteq\D$ for some $a\in\D$.
  \end{itemize}
\end{fact}

\begin{proof}
  1$\Rightarrow$2. \ 
  Pick $a\in\cap\,Z{\cdot}\D$.
  Suppose for a contradiction that $g{\cdot}a\notin\D$ for some $g\in\Z$ such that $g\nonfork_M a$.
  Then $m{\cdot}a\notin\D$ for some $m\in Z$.
  This contradicts the choice of $a$.  

  2$\Rightarrow$1. \ 
  It suffices to note that $m\cdot\Z\circ a=\Z\circ a$ for every $m\in Z$.
\end{proof}

The following is the dual version of the above fact.

\begin{fact}
  Let $\D\subseteq\X$ be definable over $M$.
  Then the following are equivalent
  \begin{itemize}
    \item [1.] $\D$ is $Z$-syndetic
    \item [2.] $(\Z\circ a)\cap\D\neq\varnothing$ for every $a\in\X$.
  \end{itemize}
\end{fact}

We say that $\C\subseteq\X$ is $\Z{\circ}$-invariant if $\Z\circ\C\subseteq\C$.
Note that, when $\Z=\X$, this says that $\C$ is a left ideal.
The following fact is easy.

\begin{fact}
  Every minimal $\Z{\circ}$-invariant subset of $\X$ is of the form $\Z\circ a$ for some $a\in\X$ so, in particular, it is type-definable over $M$.
  We also have that $\Z\circ a=\M\circ a$ for $\M$ any left ideal of $\Z$.
\end{fact}

We say that $a\in\X$ is \emph{minimal\/} if $\Z\circ a$ is a minimal $\Z{\circ}$-invariant subset of $\X$.

\begin{theorem}
  Let $\D\subseteq\X$ be definable over $M$.
  Then the following are equivalent
  \begin{itemize}
    \item [1.] $\D$ is weakly $Z$-thick
    \item [2.] $a\subseteq\D$ for some minimal $a\in\D$.
  \end{itemize}
\end{theorem}

\begin{proof}
  2$\Rightarrow$1. \ 
  Let $b\in\Z\circ a$.
  Then $b\in\Z\circ\D$.
  Therefore $b\in m{\cdot}\D$ for some $m\in Z$.
  This proves that $\Z\circ a\subseteq\cup\,Z{\cdot}\D$.
  By compactness $\Z\circ a\subseteq\cup\,C{\cdot}\D$ for some finite $C\subseteq Z$.
  By Fact~\ref{fact_thick_circ}, $\D$ is weakly $Z$-thick.

  1$\Rightarrow$2. \ 
  Let $C\subseteq Z$ be a finite set such that $\cup\,C{\cdot}\D$ is $Z$-thick.
  By Fact~\ref{fact_thick_circ}, there is $a\in\cup\,C{\cdot}\D$ such that $\Z\circ a\subseteq\cup\,C{\cdot}\D$.
  Let $a'\in\Z\circ a$ be minimal.
  As $a'\in m{\cdot}\D$ for some $m\in C$, we conclude that $m^{-1}\!\cdot a'\in\D$.
  Then $m^{-1}\!\cdot a'$ is the minimal element required by (2).
\end{proof}


% \begin{fact}
%   There is bijection $f:\Z\to\X$ such that 
% \end{fact}

