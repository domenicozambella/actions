% !TEX root = actions.tex
\section{Definable actions}
\label{definable}

\def\ceq#1#2#3{\noindent\parbox[t]{15ex}{$\displaystyle #1$}\parbox{6ex}{\hfil $#2$}{$\displaystyle #3$}}

In this section $\Z$ is a group that acts on $\X$.
The group operations and the group action are assumed definable.
We use the symbol $\,\cdot\,$ for both the group multiplication and the group action.
Let $M\preceq\U$ and $\Psi\subseteq L_x(M)$.
In this section $\Delta$ contains formulas $\varphi(x\,;z)$ of the form $\psi(z^{-1}\!\cdot x)$ for $\psi(x)\in\Psi$.
The sets $\varphi(\X\,;\Z)$ are $\Z$-invariant.
%We write $1$ for the identity of $\Z$.

It is worth noticing that automorphisms of $\UDelta$ need not preserve the group operations nor the group action.

To each $h\in\Z$ we associate the $\LDelta$-automorphism $\langle a\,;g\rangle\mapsto\langle h{\cdot}a\,;h{\cdot}g\rangle$.
Therefore $\Z$ is, up to isomorphism, a subgroup of ${\rm Aut}(\UDelta)$.
Note that $g\cdot\varphi(\X\,;h)=\varphi(\X\,;g{\cdot}h)$ for all $g,h\in\Z$ and $\varphi(x\,;z)\in\Delta$.
Therefore, the orbit of $\varphi(\X\,;h)$ under the action of $\Z$ is $\{\varphi(\X\,;g) : g\in\Z\}$.
It follows that the orbit of $\varphi(\X\,;h)$ under the actions of ${\rm Aut}(\UDelta)$ coincides with the orbit under the action of $\Z$.

We write $X$ and $Z$ for $\X\cap M^x$ and $\Z\cap M^z$ respectively.
Clearly $Z\le{\rm Aut}(\UDelta/\{Z\})$, the latter is the subgroup of ${\rm Aut}(\UDelta)$ that fixes $Z$ setwise.
By the observation above, when $h\in Z$ and $\varphi(x\,;z)\in\Delta$, the orbit of $\varphi(\X\,;h)$ under the action of ${\rm Aut}(\UDelta/\{Z\})$ coincides with the orbit under the action of $Z$.
% Also, note that up to equivalence $\BDelta(Z)$ coincides with $\Delta$.

A subset $A\subseteq\X$ is dense if $\psi(A)\neq\varnothing$ for every $\psi(x)\in\Psi$.

\begin{fact}
  The following are equivalent
  \begin{itemize}
    \item [1.] $\Z$ acts transitively on $\X$
    \item [2.] $Z{\cdot}a$ is dense for every $a\in\X$.
  \end{itemize}
\end{fact}

\begin{proof}
  Elementarity gives 1$\Rightarrow$2, saturation gives $2\Rightarrow$1.
\end{proof}

\begin{fact}
  There is bijection $f:\Z\to\X$ such that 
\end{fact}

When $\C,\D\subseteq\Z$ we define

\ceq{\hfill\C*\D}{=}{\Big\{g{\cdot}h\ :\ g\in\C,\ h\in\D,\ g\nonfork_M h\text{ for every }M\preceq\UDelta%,\ 1\in M^z
\Big\}}

When $\C\subseteq\Z$ and $\D\subseteq\X$ the definition of $\C*\D$ is formally the same as the above and we also define

\ceq{\hfill\C\circ\D}{=}{\Big\{g{\cdot}h\ :\ g\in\C,\ h\in\D,\ g\nonfork_M\D\text{ for every }M\preceq\UDelta%,\ 1\in M^z
\Big\}}



